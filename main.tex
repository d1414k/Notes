%% start of file `template.tex'.
%% Copyright 2006-2013 Xavier Danaux (xdanaux@gmail.com).
%
% This work may be distributed and/or modified under the
% conditions of the LaTeX Project Public License version 1.3c,
% available at http://www.latex-project.org/lppl/.


\documentclass[11pt,a4paper,roman]{moderncv}        % possible options include font size ('10pt', '11pt' and '12pt'), paper size ('a4paper', 'letterpaper', 'a5paper', 'legalpaper', 'executivepaper' and 'landscape') and font family ('sans' and 'roman')

% modern themes
\moderncvstyle{banking}                            % style options are 'casual' (default), 'classic', 'oldstyle' and 'banking'
\moderncvcolor{blue}                                % color options 'blue' (default), 'orange', 'green', 'red', 'purple', 'grey' and 'black'
%\renewcommand{\familydefault}{\sfdefault}         % to set the default font; use '\sfdefault' for the default sans serif font, '\rmdefault' for the default roman one, or any tex font name
\nopagenumbers{}                                  % uncomment to suppress automatic page numbering for CVs longer than one page

% character encoding
\usepackage[utf8]{inputenc}
\usepackage{fontawesome}
\usepackage{tabularx}
\usepackage{ragged2e}
% if you are not using xelatex ou lualatex, replace by the encoding you are using
%\usepackage{CJKutf8}                              % if you need to use CJK to typeset your resume in Chinese, Japanese or Korean

% adjust the page margins
\usepackage[scale=0.8]{geometry}
\usepackage{multicol}
%\setlength{\hintscolumnwidth}{3cm}                % if you want to change the width of the column with the dates
%\setlength{\makecvtitlenamewidth}{10cm}           % for the 'classic' style, if you want to force the width allocated to your name and avoid line breaks. be careful though, the length is normally calculated to avoid any overlap with your personal info; use this at your own typographical risks...

\usepackage{import}

% personal data
\name{Deepak Kumar}{Vishwakarma}
% \title{Curriculum Vitae}                               % optional, remove / comment the line if not wanted
\address{Samsung R\&D Noida }{}{}% optional, remove / comment the line if not wanted; the "postcode city" and and "country" arguments can be omitted or provided empty
% \phone[mobile]{909-839-3097}                   % optional, remove / comment the line if not wanted
% \phone[fixed]{01234 123456}                    % optional, remove / comment the line if not wanted
%\phone[fax]{+3~(456)~789~012}                      % optional, remove / comment the line if not wanted
% \email{xpan1@swarthmore.edu}                               % optional, remove / comment the line if not wanted
% \homepage{shawnpan.me}                         % optional, remove / comment the line if not wanted
% \extrainfo{}                 % optional, remove / comment the line if not wanted
%\photo[64pt][0.4pt]{picture}                       % optional, remove / comment the line if not wanted; '64pt' is the height the picture must be resized to, 0.4pt is the thickness of the frame around it (put it to 0pt for no frame) and 'picture' is the name of the picture file
%\quote{Some quote}                                 % optional, remove / comment the line if not wanted

% to show numerical labels in the bibliography (default is to show no labels); only useful if you make citations in your resume
%\makeatletter
%\renewcommand*{\bibliographyitemlabel}{\@biblabel{\arabic{enumiv}}}
%\makeatother
%\renewcommand*{\bibliographyitemlabel}{[\arabic{enumiv}]}% CONSIDER REPLACING THE ABOVE BY THIS

% bibliography with mutiple entries
%\usepackage{multibib}
%\newcites{book,misc}{{Books},{Others}}
  
\newcommand*{\customcventry}[7][.25em]{
  \begin{tabular}{@{}l} 
    {\bfseries #4}
  \end{tabular}
  \hfill% move it to the right
  \begin{tabular}{l@{}}
     {\bfseries #5}
  \end{tabular} \\
  \begin{tabular}{@{}l} 
    {\itshape #3}
  \end{tabular}
  \hfill% move it to the right
  \begin{tabular}{l@{}}
     {\itshape #2}
  \end{tabular}
  \ifx&#7&%
  \else{\\%
    \begin{minipage}{\maincolumnwidth}%
      \small#7%
    \end{minipage}}\fi%
  \par\addvspace{#1}}

\newcommand*{\customcvproject}[4][.25em]{
%   \vfill\noindent
  \begin{tabular}{@{}l} 
    {\bfseries #2}
  \end{tabular}
  \hfill% move it to the right
  \begin{tabular}{l@{}}
     {\itshape #3}
  \end{tabular}
  \ifx&#4&%
  \else{\\%
    \begin{minipage}{\maincolumnwidth}%
      \small#4%
    \end{minipage}}\fi%
  \par\addvspace{#1}}

\setlength{\tabcolsep}{12pt}

%----------------------------------------------------------------------------------
%            content
%----------------------------------------------------------------------------------
\begin{document}
%\begin{CJK*}{UTF8}{gbsn}                          % to typeset your resume in Chinese using CJK
%-----       resume       ---------------------------------------------------------
\makecvtitle
\vspace*{-23mm}

\begin{center}
\begin{tabular}{ c c c c }
  \faEnvelopeO\enspace deepakkumarv128@gmail.com& \faGithub\enspace \href{https://github.com/d1414k}{d1414k}   &  \faMobile\enspace +91-7533978164\\  
\end{tabular}
\end{center}

\section{EDUCATION}
{\customcventry{May 2019}{B-Tech in Computer Science and Engineering     CGPA: 9.16/10.00 }{National Institute of Technology, Uttarakhand}{Uttarakhand, In}{}{}}


\section{EXPERIENCE}

{\customcventry{June 2019 - Present}{Software Engineer}{Samsung R\&D Institute }{Noida, In}{}
{\begin{itemize}
\smallskip
  \item Establish development standard, improve development process and support to follow development schedule and help achieve R\&D projects objectives.
  \smallskip
  \item Revised implementation for \textbf{WiFi Direct} and \textbf{WiFi Aware} features of Samsung mobiles.
  \smallskip
  \item Enhancement of existing features and add new features as per requirements.
  \smallskip
  \item Worked on binary upgrade for different devices \textbf{i.e.}- \textbf{A71, M51, S10, S7+}
  \smallskip
\end{itemize}
}
}


\section{PROJECTS}

{\customcvproject{Wireless Intercom for Android devices}{\textit{Android, NSD}}
{\begin{itemize}
  \item This is Samsung internal project where one user can call to other user connected in same network with WiFi without internet connection.
  \smallskip
  \item In this feature I used \textbf{Network service discovery} protocol that provides the API to discover services on a network also detection of networks by identifying resources.
\end{itemize}
}

{\customcvproject{Transferring of files using WiFi Direct and Aware Technology }{\textit{Android, Java, WiFi}}
{\begin{itemize}
  \item This is also Samsung internal project where multiple android devices connected using Wi-Fi Direct (\textbf{P2P}) technology and then transfer files using socket programming.
  \smallskip
  \item Also when two or more devices connected and form a cluster i.e Wi-Fi aware (\textbf{NAN}) cluster and then they can communicate via this.
  \smallskip
\end{itemize}
}
}

{\customcvproject{Training and Placement Cell}{\textit{Html5, JSP, MySql}}
  {\begin{itemize}
    \item Renovate the website of Training placement Cell of NIT Uttarakhand more dynamically.
    \smallskip
    \item Manage website infrastructure and developed more responsive web pages.
    \smallskip
  \end{itemize}
  }
}

\section{Skills}

\item \textbf{Programming Languages}\\
 C, Java

 \item \textbf{Platforms and Tools}\\
 Windows, JVM, NetBeans, Perforce, Wireshark, Android Studio, Git

 \item \textbf{IT Constructs}\\
 Data Structures and Algorithms, OOPs, SQL, Operating System and DBMS

\section{ACHIEVEMENTS}
\begin{minipage}{\maincolumnwidth}%
	\small{
    	\begin{itemize}
          \item Cleared \textbf{SWC Professional Exam} in Samsung Electronics Pvt. Ltd.
          \item Secured \textbf{AIR-788} in \textbf{GATE-2019.}
          \item Winner of CodeByte'18 in NIT-UK
		\end{itemize}}%
\end{minipage}%
      
}
% Publications from a BibTeX file without multibib
%  for numerical labels: \renewcommand{\bibliographyitemlabel}{\@biblabel{\arabic{enumiv}}}% CONSIDER MERGING WITH PREAMBLE PART
%  to redefine the heading string ("Publications"): \renewcommand{\refname}{Articles}
\nocite{*}
\bibliographystyle{plain}
\bibliography{publications}                        % 'publications' is the name of a BibTeX file

% Publications from a BibTeX file using the multibib package
%\section{Publications}
%\nocitebook{book1,book2}
%\bibliographystylebook{plain}
%\bibliographybook{publications}                   % 'publications' is the name of a BibTeX file
%\nocitemisc{misc1,misc2,misc3}
%\bibliographystylemisc{plain}
%\bibliographymisc{publications}                   % 'publications' is the name of a BibTeX file

%-----       letter       ---------------------------------------------------------

\end{document}


%% end of file `template.tex'.
